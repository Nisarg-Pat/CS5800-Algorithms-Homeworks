%--------- Student instruction: change this by adding your neu id into the {}
\def\yourname{patel.nisargs}
%-------------------------------------------------------------------------------------------------------------

%% ================= no need to edit any of this stuff
% --- no need to change anything in this section -----------------------------------------------------
\def\homework{1} % 0 for solution, 1 for problem-set only
\def\duedate{fri apr 15, 2022 at 11.59p}
\def\duelocation{via \href{https://gradescope.com/courses/331917}{gradescope}}
\def\hnumber{8}
\def\prof{abhi shelat}
\def\course{\href{https://shelat.khoury.neu.edu/22s-5800}{cs5800 algorithms s'22}}

\documentclass[11pt]{article}
%%% ==== standard installations of latex include all of the files that are referenced in this section.  However,
%%% ==== if you are having compile problems, consider commenting some of these commands out 
\usepackage[colorlinks,urlcolor=blue]{hyperref}
\usepackage[osf]{mathpazo}
\usepackage{amsmath,amsfonts,graphicx}
\usepackage{latexsym}
\usepackage[top=1in,bottom=1.3in,left=1.5in,right=1.5in,centering]{geometry}
\usepackage{color}
\usepackage{clrscode}
\definecolor{mdb}{rgb}{0.3,0.02,0.02} 
\definecolor{cit}{rgb}{0.05,0.2,0.45} 
\markboth{\yourname}{\yourname}
%%% ===================================================================


%%% ============ should be no need to edit anything in this section ====================
\newenvironment{proof}{\par\noindent{\it Proof.}\hspace*{1em}}{$\Box$\bigskip}
\newcommand{\qed}{$\Box$}
\newcommand{\alg}[1]{\mathsf{#1}}
\newcommand{\handout}{
   \renewcommand{\thepage}{H\hnumber-\arabic{page}}%
   \noindent%
   \begin{center}%
      \vbox{%
    \hbox to \columnwidth {\sc{\course} --- abhi shelat \hfill}%
    \vspace{-2mm}%
    \hbox to \columnwidth {\sc due \MakeLowercase{\duedate} \duelocation\hfill {\Huge\color{mdb}H\hnumber.\yourname}}%
      }
   \end{center}
   \vspace*{2mm}
}
\newcommand{\solution}[1]{\medskip\noindent\textbf{Solution:}#1}
\newcommand{\bit}[1]{\{0,1\}^{ #1 }}
\newtheorem{problem}{\sc\color{cit}problem}
\newtheorem{lemma}{Lemma}
\newtheorem{definition}{Definition}
%%% ===================================================================
\thispagestyle{empty}

\begin{document}\handout


\begin{problem}Number of shortest paths\end{problem}
Given a graph $G=(V,E)$ with unit edge weights, and a starting node $s$, let $\delta(s,v)$ be the length of the shortest path between $s$ and $v$ (i.e., the smallest number of edges between $s$ and $v$).

Define a new variable $n_{v}$ which records the number of distinct shortest paths from $s$ to $v$ that have length $\delta(s,v)$ for every node in $V$.  Design an algorithm that computes $n_v$ for all nodes in $V$. Analyze the running time of the algorithm, and provide one-two sentence explanation for why the algorithm works.

\hfill

\noindent\textbf{Solution:}
We can solve this problem by modifying Breadth-First Search algorithm as follows:
\begin{codebox}
    \Procname{$\proc{CountShortestPaths}(V,E,s)$}
    \li \textbf{for} each $u \in V-\{s\}$
    \li \quad $(d_u, n_u) \leftarrow (\infty, 0)$
    \li $(d_s, n_s) \leftarrow (0, 1)$
    \li $Q \leftarrow \phi$
    \li $\proc{Enqueue}(Q,s)$
    \li \textbf{while} $Q \neq \phi$
    \li \quad $u \leftarrow \proc{Dequeue}(Q)$
    \li \quad \textbf{for} each $v \in Adj(u)$
    \li \quad \quad \textbf{if} $d_v > d_u + 1$
    \li \quad \quad \quad $(d_v, n_v) \leftarrow (d_u + 1, n_u)$
    \li \quad \quad \quad $\proc{Enqueue}(Q,v)$
    \li \quad \quad \textbf{else if} $d_v = d_u + 1$
    \li \quad \quad \quad $n_v \leftarrow n_v+n_u$
    \li \textbf{return} $n$
\end{codebox}

\textbf{Time Complexity Analysis:}\\
The running time complexity remains same as of BFS.\\
Time Complexity of $\proc{CountShortestPaths}(V,E,s)$ is $\Theta (V+E)$.\\

\textbf{Correctness:}\\
Let $p$ be any shortest path from $s$ to $v$ then there exists a node $u$ adjecent to $v$ which is before $v$ in $p$($s \rightarrow \ldots \rightarrow u \rightarrow v$) such that $\delta(s,v) = \delta(s,u)+1$. Thus adding distinct shortest paths of each $u$ satisfying the above condition, gives the distinct shortest paths for $v$.
\newpage
\begin{problem}Sparse graphs and short paths\end{problem}

\noindent Let $G=(V,E)$ be a directed graph with edge weights $w(e)$ and no negative cycles.

\begin{enumerate}

\item State the run time of the All-pairs shortest path algorithm discussed in class.


\item Consider the following algorithm.

\begin{codebox}
\Procname{$\proc{AnotherShortest}(G,w)$}
\li Add a new node $s'$ to $G$. Add edges of weight $0$ from $s'$ to every vertex $v\in V$.\\
Call this new graph $G'$.
\li Run $\proc{BellmanFord}(G',s')$ to produce shortest path lengths $\delta(s',v)$.\\
If shortest paths are not well-defined, than halt.
\li For each $e=(x,y)\in E$, set $w'(e) \gets w(e) + \delta(s',x) - \delta(s',y)$
\li For each $u\in V$, run $\proc{Dijkstra}(G,u,w')$ to compute $\delta(u,v)$ for all $v\in V$.
\li Set $d_{u,v} \gets \delta(u,v) - \delta(s',u) + \delta(s',v)$
\end{codebox}

\noindent This problem will analyze what this algorithm does and why it works.  The first step is to argue that the new edge weights $w'$ that are defined in step (3) are all non-negative.

Prove that for all $e\in E$, $w'(e) \geq 0$.


\item This explains why we can use the fast $\proc{Dijkstra}$ algorithm with edge weight $w'$ in step (4) to compute shortest paths from node $v\in V$ to all other nodes in the graph.  However, we must argue that the shortest paths under $w'$ and under $w$ will be the same shortest path.

 Prove that for any pairs of nodes $u,v\in V$, if $p$ is a shortest path from $u$ to $v$
with respect to edge weight function $w'$, then $p$ is also a shortest path from $u$ to $v$ with respect to
edge weight function $w$.


\item  What is the running time of $\proc{AnotherShortest}$ in terms of $V$ and $E$?  When
does this algorithm run faster than the All-pairs algorithm discussed in class?
\end{enumerate}

\noindent\textbf{Solution:}
\begin{enumerate}
    \item Floyd-Warshall All-pairs shortest path has time complexity $\Theta(V^3)$.
    
    \item For any edge $e(x,y)$, applying triangle inequality, we have:
        \begin{equation}\label{Q2_1}
            \begin{split}
                \delta(s',y) \leq \delta(s',x) + w(e)\\
                \delta(s',x) + w(e) - \delta(s',y) \geq 0\\
                w'(e) \geq 0
            \end{split}
        \end{equation}
        Thus for all $e\in E$, $w'(e) \geq 0$.
        
    \item For any path $p = (u,x_1, x_2,...,x_n, v)$ computed on $w'$, we have:
    \begin{equation}\label{Q2_2}
        \begin{split}
            w'(p) = \delta(u,v) &= w'(u, x_1)+w'(u, x_2)+...+w'(x_n, v)\\
                &= w(u, x_1)+\delta(s', u)-\delta(s', x_1)\\
                &\quad +w(x_1, x_2)+\delta(s', x_1)-\delta(s', x_2)\\
                &\quad +\ldots\\
                &\quad +w(x_n, v)+\delta(s', x_n)-\delta(s', v)\\
                w'(p) &= w(p)+\delta(s', u)-\delta(s', v)
        \end{split}
    \end{equation}
    
    Let $p$ be the shortest path between $u$ and $v$ calculated on $w'$. Let $p*$ be the shortest path between $u$ and $v$ computed on $w$. Then we have,
    \begin{equation}\label{Q2_3}
        \begin{split}
            w(p*) &\leq w(p)
        \end{split}
    \end{equation}
    And,
    \begin{equation}\label{Q2_4}
        \begin{split}
            w'(p) &\leq w'(p*)\\
            w(p)+\delta(s', u)-\delta(s', v) &\leq w(p*)+\delta(s', u)-\delta(s', v)\\
            w(p) &\leq w(p*)
        \end{split}
    \end{equation}
    
    Thus, from \eqref{Q2_3} and \eqref{Q2_4}, we have $w(p) = w(p*)$. So $p$ is also the shortest path between $u$ and $v$ computed on $w$. Hence proved.
    
    \item \textbf{Time Complexity Analysis of $\proc{AnotherShortest}(G,w)$:}
    \begin{enumerate}
        \item Line $1$ takes $\Theta(V)$. The new graph $G'$ has $V'=V+1$ vertices and $E'=E+V$ edges.
        \item Line $2$ runs $\proc{BellmanFord}(G',s')$ in $\Theta(V(E+V))$.
        \item Line $3$ runs in $\Theta(E)$.
        \item Line $4$ runs $\proc{Dijkstra}(G,u,w')$ for $V$ nodes, runtime: $\Theta(V(V+E)\log(V))$.
        \item Line $5$ runs in $\Theta(V^2)$.
    \end{enumerate}
    The overall time complexity is predominated by line $4$, $\Theta(V(V+E)\log(V))$.\\
    
    \textbf{Comparision with Floyd–Warshall:}\\
    This algorithm will have same time complexity as Floyd-Warshall when $E$ is $\Theta(V^2/\log(V))$. Thus it will be faster for $E = \Theta(1)$ or $\Theta(V)$. Hence this algorithm works better on sparse graphs.

\end{enumerate}


\newpage

\begin{problem} Edmonds-Karp shortest paths\end{problem}
In class, we stated that in the Edmonds-Karp maxflow algorithm, the length of
shortest paths in $G$ are monotonically increasing.  However, this is not obvious because as we add augmenting paths, new edges are introduced to the graph. In this problem, we will prove the following:

\begin{lemma}
For any $j>i$ and for any $u\in V$,  $\delta_j(s,u) \geq \delta_i(s,u)$.  
\end{lemma}
%

\noindent The proof will be by contradiction.
Suppose not, for the sake of contradiction.
Let $i$ be the first time that the shortest path distance to some node decreases after pushing flow along the $i$\textsuperscript{th} augmentation.
Moreover, let $v$ be the \emph{node with the smallest} distance to $s$ at $i+1$ for which  $\delta_i(s,v) > \delta_{i+1}(s,v)$. Let $p_i,p_{i+1}$ be respective shortest paths from $s$ to $v$ at times $i$ and $i+1$.

\medskip

Each answer should be roughly 1 sentence. You may refer to steps (1)--(7) in your explanations.

\begin{enumerate}
    \item Define node $u$ to be the node that occurs before $v$ on path $p_{i+1}$.  The first claim is that $\delta_{i+1}(s,u) \geq \delta_i(s,u)$.
    Why does this follow? (one sentence)
    
    \item Next, explain why $\delta_{i+1}(s,v) = \delta_{i+1}(s,u) + 1$. 
    
    \item Explain why edge $e_{i+1}=(u, v)$ did not exist in the graph at time $i$. 
    
    \item Thus, the edge $e_{i+1}$ must have been added after the $i$ flow, which implies that the augmenting path at $i$ took the form $s\leadsto v \rightarrow u \leadsto t$ , i.e., that pushed flow from $v$ to $u$.
    %    
    Explain why at time $i$, we have
    $$\delta_{i}(s,u) = \delta_{i}(s,v)+1$$

    \item Explain why (4) implies
    $$\delta_{i+1}(s,u) \geq \delta_i(s,v)+1$$
    %
    \item Adding one to each side, we have
    $$\delta_{i+1}(s,u) +1 \geq \delta_i(s,v)+2$$
    Explain why this implies that
    $$\delta_{i+1}(s,v) \geq \delta_i(s,v)+2$$
    \item Explain why the last statement in (6) is a contradiction.
\end{enumerate}
\newpage
\noindent\textbf{Solution:}

\begin{enumerate}
    \item Since at $i+1$, $v$ is the node with smallest distance to $s$ for which $\delta_{i+1}(s,v) < \delta_{i}(s,v)$, and $u$ occurs before $v$ on path $p_{i+1}$, then
    $$\delta_{i+1}(s,u) \geq \delta_i(s,u)$$
    
    \item As $u$ is the node that occurs just before $v$ on shortest path $p_{i+1}$,
    $$\delta_{i+1}(s,v) = \delta_{i+1}(s,u) + 1$$
    
    \item Now from the given condition, we have:
        \begin{equation}\label{Q3_1}
            \begin{split}
                \delta_{i}(s,v) &> \delta_{i+1}(s,v)\\
                \delta_{i}(s,v) &> \delta_{i+1}(s,u)+1 \quad \text{From 2}\\
                \delta_{i}(s,v) &> \delta_{i}(s,u)+1 \quad \text{From 1}\\
            \end{split}
        \end{equation}
        If edge $e_{i+1}=(u, v)$ exist at time $i$, then $\delta_{i}(s,v) \leq \delta_{i}(s,u)+1$ based on triangle inequality which contradicts the above result. Hence edge $e_{i+1}=(u, v)$ does not exist at time $i$. 
        
    \item Based on result of $3$, edge $e_{i+1}=(u, v)$ must be added after the $i$ flow, and the augmenting path at $i$ took $s\leadsto v \rightarrow u \leadsto t$.Thus $v$ comes before $u$ at time $i$. Hence,
        $$\delta_{i}(s,u) = \delta_{i}(s,v)+1$$
        
    \item Since, $\delta_{i+1}(s,u) \geq \delta_i(s,u)$, (From 1). Substituting 5 in it, we have:
    $$\delta_{i+1}(s,u) \geq \delta_{i}(s,v)+1$$
    
    \item 
        \begin{equation}\label{Q3_2}
            \begin{split}
                \delta_{i+1}(s,u) &\geq \delta_{i}(s,v)+1 \quad \text{From 5}\\
                \delta_{i+1}(s,u)+1 &\geq \delta_{i}(s,v)+2\\
                \delta_{i+1}(s,v) &\geq \delta_{i}(s,v)+2 \quad \text{From 2}
            \end{split}
        \end{equation}
    
    \item The last equation in 6 contradicts our initial assumption: $\delta_{i+1}(s,v) < \delta_{i}(s,v)$.\\
    Thus for every node $v$, we have $\delta_{i+1}(s,v) \geq \delta_{i}(s,v)$ for any time $i$.\\
    Let $j > i$, then we have $\delta_{j}(s,v) \geq \delta_{j-1}(s,v) \geq ... \geq \delta_{i+1}(s,v) \geq \delta_{i}(s,v)$.\\
    Hence our lemma holds.
    
\end{enumerate}

\end{document}
