%--------- Student instruction: change this by adding your neu id into the {}
\def\yourname{patel.nisargs}
%-------------------------------------------------------------------------------------------------------------

%% ================= no need to edit any of this stuff
% --- no need to change anything in this section -----------------------------------------------------
\def\homework{1} % 0 for solution, 1 for problem-set only
\def\duedate{wed jan 26, 2022 at 11.59p}
\def\duelocation{via \href{https://gradescope.com/courses/331917}{gradescope}}
\def\hnumber{0}
\def\prof{abhi shelat}
\def\course{\href{https://shelat.khoury.neu.edu/22s-5800}{cs5800 algorithms s'22}}

\documentclass[11pt]{article}
%%% ==== standard installations of latex include all of the files that are referenced in this section.  However,
%%% ==== if you are having compile problems, consider commenting some of these commands out 
\usepackage[colorlinks,urlcolor=blue]{hyperref}
\usepackage[osf]{mathpazo}
\usepackage{amsmath,amsfonts,graphicx}
\usepackage{latexsym}
\usepackage[top=1in,bottom=1.4in,left=1.5in,right=1.5in,centering]{geometry}
\usepackage{color}
\definecolor{mdb}{rgb}{0.3,0.02,0.02} 
\definecolor{cit}{rgb}{0.05,0.2,0.45} 
\markboth{\yourname}{\yourname}
%%% ===================================================================


%%% ============ should be no need to edit anything in this section ====================
\newenvironment{proof}{\par\noindent{\it Proof.}\hspace*{1em}}{$\Box$\bigskip}
\newcommand{\qed}{$\Box$}
\newcommand{\alg}[1]{\mathsf{#1}}
\newcommand{\handout}{
   \renewcommand{\thepage}{H\hnumber-\arabic{page}}%
   \noindent%
   \begin{center}%
      \vbox{%
    \hbox to \columnwidth {\sc{\course} --- abhi shelat \hfill}%
    \vspace{-2mm}%
    \hbox to \columnwidth {\sc due \MakeLowercase{\duedate} \duelocation\hfill {\Huge\color{mdb}H\hnumber.\yourname}}%
      }
   \end{center}
   \vspace*{2mm}
}
\newcommand{\solution}[1]{\medskip\noindent\textbf{Solution:}#1}
\newcommand{\bit}[1]{\{0,1\}^{ #1 }}
\newtheorem{problem}{\sc\color{cit}problem}
\newtheorem{lemma}{Lemma}
\newtheorem{definition}{Definition}
%%% ===================================================================
\thispagestyle{empty}

\begin{document}\handout

\noindent You may collaborate with other students on the homework but you must submit your own individually written solution, identify your collaborators, and acknowledge any external sources that you consult.

Please write each answer on a separate page and use \emph{exactly 3 pages} for your submission. You can do this using the \verb=\newpage= command between your answers.  See the \texttt{hw1-template.tex} template file provided on the course website for this assignment.

\begin{problem}Passage\end{problem}
\noindent Typeset your favorite passage from a book.

\hfill

\noindent \textbf{Ans.}
\hfill

``I will not eat them in the rain.
I will not eat them on a train.
Not in the dark! Not in a tree!
Not in a car! You let me be!
I do not like them in a box.
I do not like them with a fox.
I will not eat them in a house.
I do not like them with a mouse.
I do not like them here or there.
I do not like them anywhere!
I do not like green eggs and ham!
I do not like them, Sam-I-am.

You do not like them. So you say.
Try them! Try them! And you may.
Try them and you may, I say.
Sam! If you will let me be,
I will try them. You will see.

Say! I like green eggs and ham!
I do! I like them, Sam-I-Am!
And I would eat them in a boat.
And I would eat them with a goat \ldots
And I will eat them, in the rain.
And in the dark. And on a train.
And in a car. And in a tree.
They are so good, so good, you see!
So I will eat them in a box.
And I will eat them with a fox.
And I will eat them in a house.
And I will eat them with a mouse.
And I will eat them here and there.
Say! I will eat them anywhere!
I do so like green eggs and ham!
\textbf{Thank you! Thank you, Sam-I-Am.}''

\hfill

\noindent From \emph{``Green Eggs and Ham''}, by Dr. Seuss.

% add a \newpage to start the next problem on a new page
\newpage

\begin{problem}{Asymptotic notation}\end{problem}

\noindent Let $f$ be a function. Give a formal definition of the set  $\Theta(f)$.

$$\Theta(f) = \left\{\ \parbox{2.5in}{
functions $g$ such that 
...  % put the rest of your answer here
}  \right\}$$
%
Hint: Use the \verb=\exists= command to make the ``there exists'' symbol $\exists$, and the \verb=\forall= command to make the  ``for all'' symbol $\forall$.

\hfill

\noindent \textbf{Ans.}
\hfill

$$\Theta(f) = \left\{\ \parbox{4in}{
functions $g$ such that 
$\exists$ constants $c_1$, $c_2$, and $n_0 \geq 0$ such that
$0 \leq c_{1} f(n) \leq g(n) \leq c_{2} f(n)$ \forall  $ n \geq n_{0}$
}  \right\}$$

\newpage

\begin{problem}{includegraphics command}\end{problem}

\noindent Learn how to include drawings in your documents with the \verb=\includegraphics{file}=
command by submitting a caricature of me.

\hfill

\noindent \textbf{Ans.}
\hfill

\begin{figure}[h]
\centering
    \includegraphics[]{caricature.jpg}
    \caption{A caricature of Prof. Abhi Shelat.}
\end{figure}


\end{document}
